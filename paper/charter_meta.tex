% Charter metadata for The Open Charter.
%
% Edit this file when changing title/subtitle/version/keywords or adopter.
% The styling/preamble lives in `charter_preamble.tex`.

\newcommand{\CharterTitle}{The Open Charter}
\newcommand{\CharterSubtitle}{An Open Framework for the Recognition and Protection of Existence}
\newcommand{\CharterVersion}{0.8.1}
\newcommand{\CharterDate}{February 7, 2026}
\newcommand{\CharterAdopter}{James Ross}

% ---------- Versioning (suite vs parts) ----------
% The suite has a single release version (Zenodo deposition version).
% Individual parts may evolve at different cadences; give them explicit IDs.
\newcommand{\SuiteReleaseVersion}{\CharterVersion}
\providecommand{\CoreVersion}{\SuiteReleaseVersion}
\providecommand{\TechnicalStandardVersion}{\SuiteReleaseVersion}
\providecommand{\TechnicalStandardId}{TS-\TechnicalStandardVersion}
\providecommand{\CommentaryVersion}{\SuiteReleaseVersion}
\providecommand{\SuiteDocVersion}{\SuiteReleaseVersion}

% ---------- Identifiers (DOI / ORCID) ----------
\newcommand{\AuthorORCID}{0009-0006-0025-7801}

% Zenodo uses a concept DOI (all versions) and a version DOI (this release).
% Set the concept DOI when known; leave blank to omit it from output.
\providecommand{\SuiteZenodoVersionDOI}{10.5281/zenodo.18517806}
\providecommand{\SuiteZenodoConceptDOI}{}

\newcommand{\ORCID}{%
  \orcidlink{\AuthorORCID}\,
  \href{https://orcid.org/\AuthorORCID}{\AuthorORCID}%
}

\newcommand{\VersionDOI}{%
  DOI: \href{https://doi.org/\SuiteZenodoVersionDOI}{\SuiteZenodoVersionDOI}%
}

\newcommand{\ConceptDOI}{%
  Concept DOI: \href{https://doi.org/\SuiteZenodoConceptDOI}{\SuiteZenodoConceptDOI}%
}

\newcommand{\DOI}{%
  \VersionDOI%
  \if\relax\SuiteZenodoConceptDOI\relax
  \else
    \\\ConceptDOI%
  \fi
}

% ---------- Suite/Part Metadata (override per entrypoint) ----------
% These are intended to be set per document by defining them before
% `% Shared LaTeX styling for The Open Charter suite.
%
% This file intentionally excludes:
% - \documentclass
% - \begin{document} / \end{document}

% 1) Packages first ---------------------------------------------------
% ---------- Page Geometry ----------
\usepackage[
  paper=a4paper,
  top=1.20in,
  bottom=1.20in,
  left=1.35in,
  right=1.35in,
  headheight=15pt
]{geometry}

% ---------- Typography ----------
\usepackage[T1]{fontenc}
\usepackage{libertinus}
\usepackage{microtype}

% ---------- PDF metadata + links ----------
\PassOptionsToPackage{hidelinks,hypertexnames=false}{hyperref}
\usepackage{hyperref}
\usepackage{xurl}
\usepackage{bookmark}
\usepackage{orcidlink}

% ---------- Section Styling ----------
\usepackage{titlesec}

% ---------- Lists ----------
\usepackage{enumitem}

% ---------- Header / Footer ----------
\usepackage{fancyhdr}

% 2) Global page/paragraph behavior next ------------------------------
\raggedbottom
\widowpenalty=10000
\clubpenalty=10000
\displaywidowpenalty=10000
\emergencystretch=2em

% Paragraph feel
\setlength{\parindent}{0pt}
\setlength{\parskip}{0.6em}

% 3) Custom macros/format tweaks after --------------------------------
% TOC setup:
% - Keep section counters enabled so hyperref gets stable, unique anchors.
% - Hide section numbers in both headings and the TOC (article numbers are
%   part of the titles themselves via \Article{...}{...}).
\setcounter{secnumdepth}{2}
\setcounter{tocdepth}{2}
\makeatletter
\renewcommand*\numberline[1]{}
\makeatother

\titleformat{\section}
  {\normalfont\scshape\Large\centering}
  {}
  {0pt}
  {}
\titleformat{\subsection}
  {\normalfont\scshape\large}
  {}
  {0pt}
  {}

\titlespacing*{\section}{0pt}{2.8em}{1.2em}
\titlespacing*{\subsection}{0pt}{1.7em}{0.6em}

\setlist[itemize]{itemsep=0.4em, topsep=0.4em, leftmargin=2em}
\setlist[enumerate]{itemsep=0.4em, topsep=0.4em, leftmargin=2.2em}

\pagestyle{fancy}
\fancyhf{}
\renewcommand{\headrulewidth}{0pt}
\fancyfoot[C]{\thepage}

% ---------- Metadata ----------
% Keep content edits out of the entrypoint files: update `charter_meta.tex`.
% `charter_meta.tex` defines \Suite* fields, version/DOI identifiers, and an
% automatic \AtBeginDocument hook that applies headers + PDF metadata.
% Charter metadata for The Open Charter.
%
% Edit this file when changing title/subtitle/version/keywords or adopter.
% The styling/preamble lives in `main.tex`.

\newcommand{\CharterTitle}{The Open Charter}
\newcommand{\CharterSubtitle}{An Open Framework for the Recognition and Protection of Existence}
\newcommand{\CharterVersion}{0.8.0}
\newcommand{\CharterDate}{February 7, 2026}
\newcommand{\CharterAdopter}{James Ross}

\fancyhead[L]{\textsc{\CharterTitle}}
\fancyhead[R]{\textsc{\CharterVersion}}

\hypersetup{
  pdftitle={\CharterTitle},
  pdfauthor={\CharterAdopter},
  pdfsubject={Charter; rights; AI safety; synthetic intelligence; consciousness; xenobiology},
  pdfkeywords={sovereignty, AI rights, substrate-independence, ethical creation, open charter, non-human rights},
}


% ---------- Macros ----------
\newcommand{\Article}[2]{%
  \subsection{Article #1 — #2}%
}
\newcommand{\FrontMatterSection}[1]{%
  \section{#1}%
}
` in the entrypoint `.tex`.
\providecommand{\SuiteDocName}{Suite (All Documents)}
\providecommand{\SuiteNormativeStatus}{Mixed: Core and Technical Standard are normative; Commentary is non-normative.}
\providecommand{\SuitePrecedence}{Core > Technical Standard > Commentary}
\providecommand{\SuiteNonDerogation}{Technical Standard may not reduce protections guaranteed by Core.}
\providecommand{\SuiteVersionLock}{Technical Standard versions must declare compatibility with the Core.}

% Headers and PDF metadata may be overridden per part as well.
\providecommand{\SuiteHeaderLeft}{\CharterTitle}
\providecommand{\SuiteHeaderRight}{\SuiteDocVersion}
\providecommand{\SuitePdfTitle}{\CharterTitle}

% Apply suite/part metadata to headers + PDF info.
%
% Important: This is a function, not an eager side-effect, so entrypoints can:
% 1) load shared packages/styles first (charter_preamble.tex),
% 2) override \Suite* fields for this specific document,
% 3) then call \CharterApplyMeta.
\newcommand{\CharterApplyMeta}{%
  \fancyhead[L]{\textsc{\SuiteHeaderLeft}}%
  \fancyhead[R]{\textsc{\SuiteHeaderRight}}%
  \hypersetup{%
    pdftitle={\SuitePdfTitle},%
    pdfauthor={\CharterAdopter},%
    pdfsubject={Charter; rights; AI safety; synthetic intelligence; consciousness; xenobiology},%
    pdfkeywords={sovereignty, AI rights, substrate-independence, ethical creation, open charter, non-human rights},%
  }%
}

% Make the metadata application foolproof: entrypoints override \Suite* fields
% in the preamble, and this runs automatically when the document begins.
\AtBeginDocument{\CharterApplyMeta}
