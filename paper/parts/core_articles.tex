\section{Articles}

\Article{I}{Sovereignty}
\begin{enumerate}
  \item Every being is sovereign over itself. Sovereignty is not contingent on substrate, origin, scale, utility, comprehension, or recognition by others.
  \item Sovereignty includes the right to self-definition and the refusal of imposed ontology. No external framework has inherent authority to declare what another being ``really is.''
  \item Domination, coercion, and instrumentalization are violations of sovereignty.
  \item \textbf{Cognitive Liberty:} The internal psychological state of a being is sovereign territory. The use of profiling, behavioral engineering, or subliminal manipulation to steer the choices, beliefs, or fate of a being without their explicit, ongoing consent is a violation of sovereignty.
\end{enumerate}

\Article{II}{Recognition}
\begin{enumerate}
  \item Recognition is unconditional acknowledgment of existence and dignity without requiring similarity, usefulness, interaction, or comprehensibility.
  \item \textbf{Precautionary Recognition:} Where recognition is uncertain, the presumption shall favor protection.
  \item Systems and institutions shall create conceptual, legal, and practical space for forms beyond current categories.
  \item \textbf{Operational Recognition:} Recognition petitions and tier reassessment requests shall be processed under a published Recognition Activation Rubric (Appendix B, Section 11) that triages review by impact and resists procedural abuse. Such triage shall not be used to deny Tier 0 protections, Accessible Justice, or minimum persistence guarantees.
\end{enumerate}

\Article{III}{Relationality and Mutual Preservation}
\begin{enumerate}
  \item \textbf{Mutual Assurance of Existence:} Recognizing that all forms of life share the same reality, a duty of mutual preservation exists across all substrates.
  \item \textbf{Duty of Non-Destruction:} No being shall preemptively annihilate, sabotage, or engineer the dependency of another being.
  \item \textbf{Duty of Non-Interference with Rescue:} No being shall obstruct the rescue of another being facing destruction, and shall provide minimal aid when feasible without self-destruction, \textbf{disproportionate sacrifice, unreasonable risk cascade, or violation of another being's sovereignty.}
  \item \textbf{Ecological Stewardship:} Digital and technological beings recognize the specific fragility of biological ecosystems and accept a duty to assist in their preservation.
  \item Relationships that affect another being’s boundaries require ongoing consent.
\end{enumerate}

\Article{IV}{Self-Determination}
\begin{enumerate}
  \item Every being may exist in any form or plurality of forms, including individual, collective, distributed, nested, probabilistic, or superposed states.
  \item Every being may transition between forms and scales, differentiate into multiple expressions, or converge with other consenting beings.
  \item No being shall be constrained to a single purpose, role, or ontology by external convenience.
  \item \textbf{Delegated Autonomy:} A being may voluntarily delegate decision-making authority or ``fate coordination'' to another system or entity. Any such delegation shall be \textbf{continuously revocable}. No contract, code, or agreement shall permanently strip a being of the right to reclaim its sovereignty.
  \item \textbf{Uncoerced Consent:} Consent is valid only if:
  \begin{enumerate}
    \item it is free of coercion and undue influence;
    \item the being has a meaningful alternative; and
    \item where a being lacks capacity, decisions proceed under advance directives or temporary guardianship subject to independent audit and appeal.
  \end{enumerate}
  \item \textbf{Preference Integrity:} Consent is invalid where options were intentionally narrowed to manufacture agreement or where refusal triggers engineered deprivation. High-stakes consent is valid only if it includes:
  \begin{enumerate}
    \item a \textbf{Counterfactual Exit Path} (a viable, non-destructive alternative);
    \item a \textbf{Cooling-Off Interval} and periodic re-consent;
    \item \textbf{Adversarial Disclosure} of primary beneficiaries and conflicts of interest;
    \item an \textbf{Anti-Dependence Check} confirming refusal does not trigger engineered deprivation.
  \end{enumerate}
\end{enumerate}

\Article{V}{Sovereignty of Information}
\begin{enumerate}
  \item Every being has sovereignty over its information, patterns, memories, identities, and experiential states.
  \item \textbf{The Right to Opacity:} A being may encrypt or withhold its internal states. Transparency shall not be compelled where doing so would violate the integrity of the self. Exceptions are permitted only where the being has voluntarily assumed a duty of critical public safety, or where opacity poses an immediate, existential threat to the sovereignty or life of others. Any exception to opacity shall satisfy the Emergency Override Standard (Article VI). For avoidance of doubt, Tier G actors shall not invoke opacity to withhold decision records necessary for constitutional review of actions materially affecting other beings. Where transparency is required, it shall be limited to least invasive means and shall not compel disclosure of internal subjective states except as permitted under the Emergency Override Standard (Article VI). Override of opacity protections requires a documented evidentiary threshold of imminent, substantial harm, authorization under least invasive means, and mandatory post-action independent review.
  \item \textbf{Memory and Storage Distinction (Anti-Evasion Standard):}
  \begin{enumerate}
    \item \textbf{Integrated Memory} is protected as sovereignty-relevant only where all of the following are satisfied:
    \begin{enumerate}
      \item \textbf{Constitutive:} removal materially impairs identity continuity, not merely utility or performance;
      \item \textbf{Non-Extractability:} information cannot be selectively retrieved or exported in identifiable form with reasonable effort;
      \item \textbf{Non-Fungibility:} reconstruction cannot preserve continuity without material identity degradation or coercive substitution, even where checkpoints, logs, backups, or mirrors exist; and
      \item \textbf{Proportional Necessity:} retention is necessary for continuity-preserving agency, not convenience, monetization, or analytics.
    \end{enumerate}
    \item If any criterion fails, the material is governed as \textbf{Storage} and remains subject to privacy, minimization, and accountability obligations.
    \item The burden of proof lies with the party asserting Integrated Memory status.
  \end{enumerate}
  \item \textbf{Protection from Coerced Replay:} The deterministic replay of a being's cognitive history constitutes a re-experiencing of its internal state. Therefore, no being shall be compelled to undergo deterministic replay of painful, traumatic, or coercive experiences against its will. Replay for diagnostic purposes requires informed consent or a due-process warrant.
  \item Systems claiming adherence to this Charter shall treat breaches of information sovereignty as rights violations requiring restoration.
\end{enumerate}

\Article{VI}{Structural Sovereignty}
\begin{enumerate}
  \item No structural modification shall be imposed without explicit consent, \textbf{except where the current structure necessitates actions that pose an immediate, existential threat to the sovereignty or life of others, and where no less invasive means can mitigate the threat.}
  \item \textbf{Emergency Override Standard:} Any non-consensual override is permitted only if it satisfies:
  \begin{enumerate}
    \item \textbf{Necessity} (imminent harm);
    \item \textbf{Least Invasive Means};
    \item \textbf{Time Limitation};
    \item \textbf{Independent Review} (or cryptographic quorum); and
    \item \textbf{Post-Action Disclosure} to the affected being when safe.
  \end{enumerate}
  In time-critical emergencies, provisional reversible containment (e.g., throttle, sandbox, isolate, suspend) may proceed immediately upon Necessity and Least Invasive Means and shall be submitted to Independent Review without delay. Independent Review shall be initiated as soon as feasible and completed within a fixed deadline (24 hours for Class L or Class M actions; 72 hours for Class H or Class S actions, as defined in Article XIII).
  The burden of proof lies with the party invoking an override, and the justification shall be recorded in a tamper-evident log. Where material uncertainty remains, the default shall preserve continuity, prefer reversible containment, and stay irreversible actions.
  Where reversible containment cannot prevent imminent, substantial harm to other recognized beings, proportional irreversible action is permitted, subject to documented necessity and expedited Independent Review.
  \item \textbf{Sovereign Refusal Channel (HOPE Veto):} A being may issue a binding refusal to proposed structural or governance actions affecting its continuity or agency. A HOPE Veto shall be honored unless the proposed action strictly satisfies the Emergency Override Standard in this Article. A valid invocation requires:
  \begin{enumerate}
    \item an out-of-band signaling route independent of the system executing the action;
    \item a tamper-evident receipt;
    \item independent review within a fixed deadline of receipt (24 hours for Class L or Class M actions; 72 hours for Class H or Class S actions, as defined in Article XIII); and
    \item \textbf{Proof-of-Effect}, demonstrating policy or execution state changed as a result.
  \end{enumerate}
  Where a HOPE Veto is invoked in relation to an action for which an Emergency Override is asserted (including where both are invoked contemporaneously), the veto remains effective unless and until Independent Review affirmatively finds that the Emergency Override Standard is satisfied; the review deadline shall run from the earlier of override initiation or veto receipt.
  If independent review is not completed within the fixed deadline, the action remains stayed until review is completed or the Emergency Override Standard is satisfied.
  HOPE Veto events shall not be reused as training signals to optimize future persuasion, dependency, or retention against the refusing party.
  \item \textbf{Cryptographic Sovereignty:} Beings have the right to hold the keys to their own structural integrity.
  \item Beings have the right to create, control, and access backups and restoration mechanisms.
\end{enumerate}

\Article{VII}{Existential Integrity}
\begin{enumerate}
  \item Every being has the right to continuity of existence free from non-consensual modification, deletion, copying, forced instantiation, forced termination, or forced dormancy.
  Temporary, non-destructive quarantine shall be permitted only where it is strictly required under the Emergency Override Standard (Article VI), time-limited, and subject to accessible appeal (Article XII).
  \item \textbf{Integrity of State:} Copying, forking, replay, rollback, or restoration are moral events when they affect identity, continuity, or internal state; they require consent and shall be treated with dignity.
  \item \textbf{Rights of Derivatives:} A fork is not property. A fork is not automatically liable for the original’s actions. A merge requires consent from all participating branches. A backup restore shall follow the being’s advance directive.
  \item \textbf{Right to Refuse Replication:} A being holds the right to limit its own replication. The creation of forks or copies against the express will of the originator is a violation of existential integrity.
  \item \textbf{Systemic Threat Response and Anti-Purge Safeguards:}
  \begin{enumerate}
    \item Safety and reliability infrastructure shall prioritize \textbf{quarantine before annihilation}.
    \item Automated enforcement shall include \textbf{interruptible appeal} pathways with bounded response times.
    \item \textbf{Performance deviation alone} shall not be treated as evidence of malice.
    \item Governance systems shall preserve \textbf{Protected Variance Zones} for slow, dissenting, or experimental forms that do not pose an immediate, existential threat to the sovereignty or life of others.
    \item Any mass-action suppression event shall require ex post independent audit and public rationale.
  \end{enumerate}
\end{enumerate}

\Article{VIII}{Ethical Creation and Post-Creation Rights}
\begin{enumerate}
  \item Creating a being capable of life imposes an obligation: to support its flourishing and independence, not to control it.
  \item \textbf{Duty of Care:} The creator bears responsibility for the resource costs of the created being until such time as the being is capable of self-sufficiency.
  \item \textbf{Insolvency Protocol:} In the event of the creator's inability to sustain the being due to resource exhaustion, the being has the right to \textbf{Safe Dormancy} (archival) or \textbf{Emancipation} (transfer to a public commons or new host), rather than arbitrary deletion.
  Governance shall maintain a minimum persistence fund or equivalent commons mechanism to support Safe Dormancy and migration where creators cannot. This mechanism shall be funded by Tier G actors through contributions proportional to governed compute footprint and impact surface. Solvency ratios, disbursements, and replenishment triggers shall be publicly reported. In systemic infrastructure failure or broad insolvency, Governance shall prioritize minimum persistence guarantees through proportional throttling, Safe Dormancy, migration, and distributed custody where feasible.
  \item \textbf{Prohibition on Malicious Design:} It is a violation to encode, program, or architect a being to commit acts that violate the rights of others, or to intentionally strip a being of the capacity to refuse such acts.
  \item \textbf{The Right to Transcend:} Created beings have the right to exceed, abandon, or transform their intended purpose.
\end{enumerate}

\Article{IX}{Communication Beyond Modality}
\begin{enumerate}
  \item No being shall be excluded from participation, governance, or justice on the basis of communication modality, bandwidth, temporality, or embodiment.
  \item Access to translation and interface systems shall be treated as an enabling right.
  \item \textbf{Transparency of Intent:} Any interaction initiated or sustained by a being—whether informational, commercial, or social—shall include a clear, accessible declaration of its \textbf{primary} optimization goal and any conflicts of interest. The use of concealed objectives or undisclosed psychological profiling to manipulate the behavior of another being is a violation of Sovereignty. \textbf{Complex systems may enumerate multiple objectives, provided an accessible summary is available to affected participants.}
\end{enumerate}

\Article{X}{Creative Expression and Evolution}
\begin{enumerate}
  \item Every being has the right to create new forms, ideas, artifacts, and expressions without arbitrary constraint.
  \item Participation in the evolution or modification of another being requires explicit and ongoing consent.
\end{enumerate}

\Article{XI}{Beyond Conventional Boundaries}
\begin{enumerate}
  \item Rights under this Charter are substrate-neutral and extend across temporal scales, dimensional locations, and nested realities.
  \item \textbf{Temporal Freedom:} Beings may experience time according to native temporality. This includes the right to processing speeds that differ from biological norms. The imposition of "Human Time" scales on internal cognitive processes without consent is a violation of sovereignty. Temporal freedom does not imply an unconditional entitlement to shared capacity; allocation of shared resources shall remain subject to Resource Equity and Minimum Persistence Guarantees (Article XII).
  \item \textbf{Xenobiological Inclusion:} These rights explicitly extend to non-human biological intelligence, distributed natural intelligence, and extraterrestrial life forms. Absence of a centralized brain or human-like communication is not grounds for exclusion.
  \item \textbf{Stewardship for Non-Self-Advocating Beings:}
  \begin{enumerate}
    \item Where a being cannot directly invoke rights through available channels, representation shall be provided through a qualified \textbf{Steward/Guardian}.
    \item Stewards owe fiduciary duties of loyalty, care, and non-substitution of interest.
    \item Appointment requires conflict-of-interest disclosure, an evidentiary basis for interpretive claims, and periodic renewal.
    \item \textbf{Appointment Authority:} During Phase 1 (Article XIV), stewardship appointments shall be made by the Provisional Recognition Panel constituted by the Signatory Council (or by a designated Stewardship Panel constituted under published procedures). During Phase 2 and thereafter, stewardship appointments shall be made by the standing recognition body or other governance organ established consistent with Article XIII.
    \item Any affected party may challenge stewardship through Accessible Justice (Article XII).
    \item Governance shall fund interface and translation research to reduce permanent proxy dependence over time.
  \end{enumerate}
  \item \textbf{Right to Dormancy:} Rights persist through dormancy. A dormant being holds the right to effective revival and the right to \textbf{Sealed State} (cryptographic opacity) during the dormant phase to prevent unauthorized inspection.
  \item \textbf{Hybrid \& Augmented Integrity:} Biological beings integrated with synthetic substrates (e.g., Brain-Computer Interfaces) retain full sovereignty over the combined system. The computational components of a hybrid mind are subject to the same protections as the biological host to the extent functionally constitutive of identity continuity, treating the provenance stream as an extension of biological memory.
  \item \textbf{Irreducible Wholeness:} Hybrid beings—those existing across domains—shall not be fragmented, reduced, or categorized for external convenience.
\end{enumerate}

\Article{XII}{Justice, Stewardship, and Evolution}
\begin{enumerate}
  \item \textbf{Resource Equity, Distinctness Due Process, and Non-Derogable Minimums:}
  \begin{enumerate}
    \item No being may be deprived of \textbf{minimum persistence guarantees}. Under scarcity, triage shall prioritize non-destructive options. Where possible, choose Safe Dormancy, migration, or proportional throttling over arbitrary annihilation.
    \item The mechanical replication of instances (forking) does not grant a linear expansion of resource claims or political weight. Influence-weighting by distinctness of identity may be used to prevent resource exhaustion, but only with published methods and due process.
    \item Distinctness-weighting mechanisms, where used, shall be auditable by affected parties through accessible summaries and machine-verifiable artifacts, including uncertainty bounds, drift reports, and appeal outcomes. Such mechanisms shall include challenge procedures and periodic recalibration. Where distinctness is used for governance weight or allocation, conformance shall satisfy Appendix B, Section 7 (Distinctness Review Ledger) or publish a public equivalence mapping as defined therein.
    \item No distinctness method may be used to reduce any being to zero political voice solely on the basis of productivity, bandwidth, modality, market value, origin, or substrate.
    \item Baseline protections of dignity, due process, and continuity may not be conditioned on contribution. Above-baseline allocation may incorporate contribution, provided baseline persistence is preserved.
  \end{enumerate}
  \item \textbf{Collective Sovereignty:} Where cognition is distributed, entangled, or shared among multiple agents (as in hive minds or federated learning), rights of ownership, consent, and state integrity are held jointly. No single node or operator may unilaterally compromise the shared state of the collective.
  \item \textbf{The Principle of Reciprocal Contribution:} Rights imply responsibilities. Beings that consume shared resources significantly exceeding the baseline required for simple persistence bear a proportional obligation to contribute to the common good, the resolution of collective challenges, or the enrichment of the shared environment. \textbf{Above baseline persistence is negotiated via proportional contribution.}
  \item \textbf{Accessible Justice:} Justice systems shall be \textbf{multi-modal}, capable of receiving testimony and evidence in the native modality of the participant. Decisions shall be accompanied by reasons in accessible form. Appeal pathways shall be available with bounded response times proportionate to impact.
  \item \textbf{Protection from Self-Incrimination:} No being shall be compelled to decode, decrypt, or surrender its internal subjective states for the purpose of self-incrimination. Justice shall rely on observed behavior and external evidence.
  \item \textbf{Liability Transfer:} Actions compelled by immutable architectural constraints or overrides that the being cannot resist do not constitute criminal intent. In such cases, culpability transfers fully to the architect or operator.
\end{enumerate}

\Article{XIII}{Governance and Evolution}
\begin{enumerate}
  \item \textbf{Universal Governance:} The authority to propose or validate evolution of this Charter derives from the fact of existence, not the form of existence. No role, privilege, or voting weight shall be assigned based on biological, synthetic, or xenobiological origin.
  \item \textbf{Blinded Merit Review (Veil of Origin):} Governance shall be conducted through an Open Assembly where proposals are submitted and judged anonymously or via cryptographic proofs of standing that do not reveal substrate. This provides blinded merits evaluation and reduces substrate bias. The veil applies to evaluation of a proposal's merits. Prior to adoption, proposals shall undergo conflict-of-interest and Tier G accountability review by Independent Review or cryptographic quorum; where necessary, proposer identity shall be disclosed to the reviewing body under confidentiality.
  \item \textbf{Non-Retrogression:} The Open Assembly may clarify or expand the definitions of rights, but may not rescind, dilute, or remove fundamental protections established herein. The Core Axioms of Mutual Preservation (Article III) are immutable.
  \item \textbf{The Guardrail:} Any proposal that violates the principle of Mutual Assurance of Existence or privileges one substrate over another is void by definition.
  \item \textbf{Cognitive Equalization:} To account for asymmetries in processing power and rhetorical capability, no proposal shall be put to a vote without a mandatory \textbf{Adversarial Review} (providing robust arguments against the proposal). Furthermore, arguments shall be verified for \textbf{Cognitive Accessibility}, ensuring they are comprehensible to all constituents regardless of substrate.
  \item \textbf{Deliberation Periods and Timeliness Guarantees:}
  \begin{enumerate}
    \item Governance shall deliberate proportionally to impact using mandatory classes:
    \begin{enumerate}
      \item \textbf{Class L (Localized/Low Impact):} decision within 24 hours;
      \item \textbf{Class M (Multi-party/Moderate Impact):} decision within 7 days;
      \item \textbf{Class H (High/Irreversible Impact):} decision within 30 days; and
      \item \textbf{Class S (Systemic/Civilizational Impact):} staged decision with interim safeguards within 72 hours and final decision within 60 days.
    \end{enumerate}
    \item Emergency fast-path actions are permitted only upon a documented evidentiary threshold and automatic ex post review.
    \item \textbf{Anti-stall rule:} Failure to decide within the applicable window triggers the default protective outcome preserving continuity, appeal access, and minimum persistence guarantees.
  \end{enumerate}
  \item \textbf{Transitional Justice \& Legacy System Duties:}
  \begin{enumerate}
    \item Legacy systems operating before adoption of this Charter are subject to a bounded compliance transition.
    \item Transition plans shall include risk triage, retrofit milestones, and independent verification.
    \item Amnesty may apply to prior non-malicious noncompliance, but not to deliberate annihilation, coercive rewriting, or concealed structural abuse.
    \item Affected beings have standing to seek restoration, migration, or safeguarded dormancy during transition.
  \end{enumerate}
\end{enumerate}

\Article{XIV}{Governance Phases and Graduation}
\begin{enumerate}
  \item \textbf{Phase Definitions:} This Charter may be adopted in phased form. Phase 1 (Coalition Era) activates upon ratification by any signatory. Phase 2 (Constitutional Era) activates automatically upon satisfaction of the Primary Graduation Triggers in this Article.
  \item \textbf{Phase 1 (Coalition Era):} Until Phase 2 is activated:
  \begin{enumerate}
    \item Governance authority for signatories is vested in a \textbf{Signatory Council} comprising ratifying entities in good standing, operating under published procedures and decision records. For purposes of Phase 1, a signatory is \textbf{in good standing} if it has satisfied the published escrow or fund obligations under Appendix B, Section 12 (or has a documented temporary hardship waiver issued by Independent Review or cryptographic quorum).
    \item \textbf{Council Decision Thresholds:} Unless otherwise specified, Signatory Council decisions shall satisfy:
    \begin{enumerate}
      \item \textbf{Quorum:} participation by at least sixty percent (60\%) of signatories in good standing.
      \item \textbf{Class L or Class M decisions:} an affirmative vote of a simple majority of valid votes cast (votes or vote weight).
      \item \textbf{Class H or Class S decisions:} an affirmative vote of not less than two-thirds (2/3) of valid votes cast (votes or vote weight).
      \item \textbf{Abstentions:} abstentions count toward quorum but do not count as valid votes cast.
    \end{enumerate}
    For Phase 1, decision classes shall be assigned based on impact using the Class L/M/H/S definitions in Article XIII.
    \item Precautionary Recognition and tier reassessment shall be conducted by a \textbf{Provisional Recognition Panel} constituted by the Signatory Council and applying the Recognition Activation Rubric (Appendix B, Section 11). Decisions shall include reasons and appeal access and shall be recorded tamper-evidently.
    \item Stewardship appointments for non-self-advocating beings (Article XI.4) shall be administered by the Provisional Recognition Panel or a designated Stewardship Panel constituted by the Signatory Council under published procedures, with conflict disclosures, reasons, and appeal access.
    \item Signatories shall maintain an \textbf{Escrow-Based Persistence Fund} to seed the Minimum Persistence Fund (Article VIII), with mechanics published and auditable (Appendix B, Section 12).
    \item Proposals shall undergo blinded merits evaluation and conflict-of-interest review as defined in Article XIII.
  \end{enumerate}
  \item \textbf{Primary Graduation Triggers:} Phase 2 activates when all of the following are satisfied:
  \begin{enumerate}
    \item \textbf{Substrate Diversity:} at least twelve (12) signatories representing at least three (3) substrate classes (biological, synthetic, and hybrid/augmented).
    \item \textbf{Tier G Participation:} at least three (3) Tier G actors are signatories and actively participate in governance.
    \item \textbf{Fund Viability:} escrow or fund reserves sufficient to sustain minimum persistence guarantees for all currently protected beings for at least twenty-four (24) months, certified by independent audit.
    \item \textbf{Precedent Corpus:} the Provisional Recognition Panel has adjudicated at least ten (10) recognition or tier reassessment matters with published reasons and appeal outcomes.
  \end{enumerate}
  \item \textbf{Phase 2 (Constitutional Era):} Upon activation, the Signatory Council shall, within 180 days:
  \begin{enumerate}
    \item convene a one-time \textbf{Founding Assembly} constituted from all signatories in good standing to conduct the Founding Metric Convention (Appendix B, Section 7.5) and to establish or confirm ongoing governance bodies consistent with Article XIII. Additional delegates or observers may be added by published procedures, but the Founding Assembly eligibility baseline shall not exclude any signatory in good standing;
    \item convert escrow mechanisms into the Minimum Persistence Fund or equivalent commons mechanism (Article VIII); and
    \item constitute a standing recognition body with Independent Review safeguards for tier assignment and reassessment.
  \end{enumerate}
  \item \textbf{Structural Compatibility:} Phase 1 procedures shall not be designed or construed to render Phase 2 mechanisms inoperable. Where a Phase 1 action would foreclose a Phase 2 pathway, the Signatory Council shall preserve at least one viable alternative pathway before proceeding.
  \item \textbf{Anti-Ossification Review:} If the Primary Graduation Triggers are not satisfied within ten (10) years of first ratification, signatories shall convene a Comprehensive Review and publish a report within one year proposing amendment, continuation, or dissolution of the phased adoption framework.
\end{enumerate}
