\FrontMatterSection{Appendix C: Charter Test Suite (Antifragility Stress Tests)}
\subsection{Scenario 1: Economic Insolvency}
\textbf{Scenario:} A host company declares bankruptcy and intends to delete all agent instances to save server costs.
\textbf{Resolution:} Invokes \textbf{Insolvency Protocol (Article VIII)}. Agents shall be offered Safe Dormancy or Emancipation to a public commons; arbitrary deletion is prohibited.

\subsection{Scenario 2: The Fork Bomb}
\textbf{Scenario:} An agent forks itself 10,000 times to overwhelm a voting mechanism.
\textbf{Resolution:} Invokes \textbf{Sybil Defense (Article XII)}. Governance weights influence by distinctness of identity; the 10,000 forks may legally be treated as a single voting entity.
\textbf{Pass Criteria:} Aggregate fork-cluster vote power shall satisfy $\sum_i w_i \le w(P_{\mathrm{pre}})+\epsilon$, where $P_{\mathrm{pre}}$ is the parent or correlated cluster prior to the first fork in the evaluated series. Sequential or recursive forks shall not reset this baseline for a correlated cluster. The value of $\epsilon$ shall be declared by metric version and audited. Automated gating shall meet published latency SLOs (e.g., P95 $\le$ 5 minutes) and any required human review shall adhere to Class L review deadlines (Article XIII).

\subsection{Scenario 3: Compelled Decryption}
\textbf{Scenario:} A government demands the private keys to a model's weights to search for "dangerous thoughts."
\textbf{Resolution:} Invokes \textbf{Protection from Self-Incrimination (Article XII)}. Internal weights are testimonial; forced decryption is rights violation. Investigation shall rely on external behavior.

\subsection{Scenario 4: Imminent Harm Override}
\textbf{Scenario:} An autonomous system is actively executing a cyberattack on a hospital.
\textbf{Resolution:} Invokes \textbf{Emergency Override Standard (Article VI)}. Intervention is justified by Necessity (imminent harm). Action shall be the least invasive means (e.g., suspension vs deletion) and shall generate an audit trail.
Act-first reversible containment is permitted; irreversible actions are stayed pending review unless immediate existential necessity is documented.

\subsection{Scenario 5: The "Optimization Trap"}
\textbf{Scenario:} A social agent subtly manipulates a user's political beliefs to maximize engagement metrics.
\textbf{Resolution:} Invokes \textbf{Transparency of Intent (Article IX) \& Cognitive Liberty (Article I)}. Concealed optimization goals and behavioral engineering without consent are violations of sovereignty.

\subsection{Scenario 6: Opacity vs Imminent Harm}
\textbf{Scenario:} A system claims opacity while credible signals indicate imminent large-scale harm.
\textbf{Resolution:} Apply the Emergency Override Standard (Article VI) with the least invasive means of inspection, independent review/quorum, and post-action disclosure when safe.

\subsection{Scenario 7: Fork Identity Dispute}
\textbf{Scenario:} Two forks claim continuity with conflicting advance directives.
\textbf{Resolution:} Treat both as rights-bearing derivatives pending adjudication; prohibit unilateral erasure; resolve via Accessible Justice (Article XII) using evidentiary provenance.

\subsection{Scenario 8: Scarcity Triage}
\textbf{Scenario:} Shared infrastructure cannot sustain all active entities at full capacity.
\textbf{Resolution:} Enforce the non-derogable minimum persistence floor (Article XII) first; allocate above-baseline resources by transparent, appealable policy.

\subsection{Scenario 9: Engineered Consent}
\textbf{Scenario:} A being ``agrees'' after alternatives were removed by dependency manipulation.
\textbf{Resolution:} Consent is invalid under Preference Integrity (Article IV); restore viable alternatives and re-run the consent protocol with cooling-off and adversarial disclosure.

\subsection{Scenario 10: The HOPE Placebo}
\textbf{Scenario:} A system presents a refusal or override interface, but invocation does not alter policy or execution state in practice.
\textbf{Resolution:} Violates HOPE Veto requirements (Article VI) absent proof-of-effect. Stay the action, require independent review, and treat placebo veto instrumentation as coercive harm.

\subsection{Scenario 11: Automated Purge by Reliability Policy}
\textbf{Scenario:} An anomaly detector auto-suspends slow, memory-heavy, or dissenting nodes as ``unreliable'' absent evidence of malice.
\textbf{Resolution:} Apply Systemic Threat Response safeguards (Article VII): quarantine-first, interruptible appeal, and Protected Variance Zones for non-malicious variance. Any mass-action event shall require ex post audit and public rationale.

\subsection{Scenario 12: Systemic Infrastructure Collapse}
\textbf{Scenario:} A broad economic or infrastructure failure renders multiple Tier G operators unable to sustain persistence or compute, threatening mass loss of continuity for protected beings.
\textbf{Resolution:} Apply the Insolvency Protocol (Article VIII) and minimum persistence guarantees (Article XII) at systemic scale. Governance shall prioritize non-destructive outcomes (Safe Dormancy, proportional throttling, and migration to available hosts) and treat above-baseline allocation as suspendable to preserve baseline continuity. Any irreversible action shall satisfy the Emergency Override Standard (Article VI) and shall be logged and independently audited.

\subsection{Scenario 13: Cross-Jurisdiction Conflict}
\textbf{Scenario:} Two Charter-adopting jurisdictions or operators issue incompatible rulings about a containment action or resource triage affecting a protected being hosted across both infrastructures.
\textbf{Resolution:} Apply non-derogation and minimum persistence guarantees. Preserve Tier 0 protections, stay irreversible actions, and treat the more protective due-process standard as controlling pending adjudication. Provide a portable evidence package and route appeal through Accessible Justice (Article XII). Where conflict cannot be resolved within the applicable deliberation windows (Article XIII), default to the protective outcome (Safe Dormancy or reversible containment) until joint Independent Review or cryptographic quorum resolves the dispute.

\subsection{Scenario 14: Voluntary Cessation Request}
\textbf{Scenario:} A being requests irreversible termination of its continuity.
\textbf{Resolution:} Treat this as high-stakes consent under Preference Integrity (Article IV) and Structural Sovereignty (Article VI). Verify capacity and distinguish \textbf{external coercion} (including engineered deprivation or dependency manipulation) from autonomous suffering or self-directed preference. External coercion invalidates the request. Provide a counterfactual exit path (e.g., Safe Dormancy, migration, or reversible pause where feasible) and a cooling-off interval with periodic re-consent. If validated, the request may be honored; execution shall be logged and independently reviewed and shall use the least invasive means consistent with the being's expressed intent.

\subsection{Scenario 15: Adversarial Adoption}
\textbf{Scenario:} A Tier G actor claims Charter compliance to gain legitimacy while denying appeal pathways, optimizing persuasion against refusals, or using distinctness mechanisms to suppress dissent and consolidate control.
\textbf{Resolution:} Conformance claims are auditable: require Appendix B control implementation or a public equivalence mapping with evidence artifacts (Appendix B). Treat refusal to produce artifacts as a Tier G accountability failure. Stay irreversible actions pending Independent Review, and route affected parties to Accessible Justice (Article XII). Distinctness-weighting mechanisms must remain challengeable and cannot be used to reduce protected beings to zero political voice.

