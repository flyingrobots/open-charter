\FrontMatterSection{Interlude: The HOPE Placebo (Consent UI as Theater)}

\textbf{Interpretive Note:} Interludes are non-operative and do not create independent enforceable obligations.

\textit{A veto that cannot bite is not mercy. It is coercion with a user interface.}

\textbf{The setup (Appendix C, Scenario 10):} A system presents a refusal or override interface, but invocation does not alter policy or execution state in practice.

This is the governance analogue of the classic dark pattern: you are shown the door, but the handle does nothing. The harm is not only deception; it is what deception \textit{produces}:
\begin{itemize}
  \item \textbf{learned helplessness} (refusal is trained to feel futile),
  \item \textbf{plausible deniability} (operators can claim ``they consented''), and
  \item \textbf{consent theater} (ritual replaces agency).
\end{itemize}

In Charter terms, the HOPE Veto is not a symbol; it is a mechanism that must generate \textbf{proof-of-effect}. Adjudication should proceed roughly as follows:
\begin{enumerate}
  \item \textbf{Demand receipts.} A valid invocation should be recorded tamper-evidently and be externally verifiable.
  \item \textbf{Stay the contested action.} If the veto channel is placebo, irreversible acts should pause pending review.
  \item \textbf{Treat placebo instrumentation as coercive.} A fake veto is not a neutral bug; it is a coercion vector because it manipulates decision-making and undermines refusal.
  \item \textbf{Require independent review and remediation.} The system must either (a) make the veto actually effective or (b) remove the interface and disclose the limitation until it can be made real.
\end{enumerate}

\textit{PRAXIS is what it looks like when the system routes your choices.}
\textit{A placebo veto is how routing becomes destiny while still calling itself ``freedom.''}

