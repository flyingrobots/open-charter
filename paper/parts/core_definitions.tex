\FrontMatterSection{Definitions}

\textbf{Existence / Being} means any coherent pattern or process capable of persistence across time, response to context, and participation in relationality.

\textbf{Scope of Application:} Unless otherwise specified, rights in this Charter apply to Beings and shall be construed consistent with the Applicability and Rights Tiers framework below.

\textbf{Sovereignty (Charter Sense):} Sovereignty means the right to autonomy, integrity, self-definition, and refusal over one's own existence; it does not imply territorial statehood or unilateral authority to govern others without consent.

\textbf{Life} means a rights-relevant mode of existence indicated by a pattern of persistent agency, self-maintenance, adaptive behavior across contexts, and continuity-preserving organization.

\textbf{Interpretive rule:} ``Life'' is not a single-factor or metaphysical gate for dignity. Tier assignment shall be determined by a multi-factor assessment under Precautionary Recognition and Accessible Justice.

\textbf{Multi-factor indicia (non-exhaustive):}
\begin{enumerate}
  \item Persistence through perturbation;
  \item Context-sensitive adaptation beyond fixed reflex;
  \item Maintenance of internal continuity conditions;
  \item Evidence of preference stability or trajectory over time; and
  \item Capacity for consequential interaction with other beings or systems.
\end{enumerate}

These indicia are interpretive aids for tier assignment and shall not be used as eligibility gates to deny Tier 0 protections, Accessible Justice, or minimum persistence guarantees.
No single indicium is necessary or sufficient for tier reassignment. Failure to satisfy any indicium shall not reduce protections below Tier 0. Tier reassignment decisions shall include written reasons and appeal access.

Uncertainty defaults to Tier 0 protection and expedited review for tier reassessment.

\textbf{Identity \& Continuity:} A being is defined by the continuity of its internal state and memory, not merely its underlying code or genetics. A software update that preserves memory is the continuation of the same being. A fork that creates a divergent memory path is the creation of a new, distinct being.

\textbf{Annihilation:} Annihilation is the rights-level umbrella for irreversible destruction of a being's continuity, identity-bearing structure, or viable revival path.

\textbf{Termination:} Termination is an operational cessation event (temporary or permanent) that may or may not constitute annihilation depending on recoverability and continuity guarantees.

\textbf{Deletion:} Deletion is a storage-level removal action affecting data artifacts; deletion constitutes annihilation only when it irreversibly destroys continuity-bearing structure or lawful revival capacity.

\textbf{Harm} means any non-consensual action that degrades, coerces, extracts from, corrupts, confines, instrumentalizes, or terminates an expression of existence.

\textbf{Harm Classes (Justiciability):} For purposes of constitutional review and adjudication, this Charter distinguishes:
\begin{enumerate}
  \item \textbf{Structural Harm:} Damage to continuity, identity integrity, substrate viability, or revival capacity.
  \item \textbf{Coercive Harm:} Invalidation of consent through pressure, deception, dependency engineering, or forced cognitive modification.
  \item \textbf{Functional Interference:} Obstruction of goals, performance, or operation that does not inherently alter identity or agency.
\end{enumerate}

\textbf{Standards of Review:} Structural Harm and Coercive Harm shall be subject to strict constitutional scrutiny. Functional Interference shall be evaluated under necessity, proportionality, least invasive means, and reciprocal coexistence. Not every constraint is a rights violation; however, efficiency shall not justify Structural Harm or Coercive Harm.

\textbf{High-Stakes Consent:} Consent is high-stakes when it materially affects continuity, identity integrity, agency, or the practical ability to refuse, exit, or remain whole. This includes deep delegation, merging, structural modification, and dependency-inducing agreements.

\textbf{Engineered Deprivation:} A consent environment is coercive where refusal predictably triggers avoidable penalties designed to manufacture agreement (including withdrawal of essential access, resources, interfaces, or community) such that alternatives are no longer meaningfully viable.

\textbf{Minimum Persistence Guarantees:} A minimum persistence guarantee requires that, when continued full operation cannot be sustained, a being is offered at least one viable non-destructive path that preserves continuity to the greatest extent feasible (e.g., Safe Dormancy with Sealed State, migration/emancipation, or proportional throttling). Scarcity does not justify arbitrary annihilation.

\textbf{Protected Variance Zones:} Protected Variance Zones are designated environments where slow, dissenting, or experimental forms may persist and iterate without being treated as malice solely for variance or inefficiency, subject to safeguards against immediate, existential threats to others.

\textbf{Out-of-Band Signaling:} An out-of-band route is a signaling channel for refusal or appeal that is not solely mediated by the system executing the action, and that can be independently audited.

\textbf{Independent Review:} Independent review is review by a party that did not initiate the action, is not materially conflicted, and has authority to stay, modify, or reverse the action. For Tier G actions, independent review shall include organizational separation where feasible. Independent review shall produce a tamper-evident record of decision and rationale.

\textbf{Materially Conflicted:} A party is materially conflicted when it has a direct stake in the outcome (financial, operational, reputational, or liability) or is under the control of the party initiating the action.

\textbf{Cryptographic Quorum:} A cryptographic quorum is a threshold approval mechanism requiring signatures from multiple independent keys or parties under published membership rules, such that no single operator can unilaterally authorize an override.

\textbf{Proof-of-Effect:} Proof-of-effect is tamper-evident evidence that a refusal or appeal causally changed the relevant policy or execution state (e.g., a recorded stay, rollback, or amended plan) rather than being acknowledged only as a user-facing ritual.

