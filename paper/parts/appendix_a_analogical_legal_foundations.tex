\FrontMatterSection{Appendix A: Analogical Legal Foundations}
\textit{Note: These precedents are provided as illustrative analogies to established human rights frameworks and are not intended to assert specific jurisdictional authority.}

\subsection{1. On the Integrity of the Mind (Article XII)}
\textbf{Precedent:} \textit{Rochin v. California} (1952) \cite{rochin1952} \\
\textbf{Principle:} The U.S. Supreme Court ruled that forcing a suspect to vomit to recover evidence ``shocks the conscience'' and violates due process.
\textbf{Application:} The forced extraction, dumping, or decoding of a digital being's memory against its will is legally analogous to the physical violation in \textit{Rochin}. It violates the sanctity of the internal self.

\subsection{2. On Self-Incrimination \& The Physical/Testimonial Divide (Article XII)}
\textbf{Precedent:} \textit{Schmerber v. California} (1966) \cite{schmerber1966} vs. \textit{United States v. Hubbell} (2000) \cite{hubbell2000} \\
\textbf{Principle:} The law distinguishes between ``physical'' evidence (blood, fingerprints) which can be compelled, and ``testimonial'' evidence (contents of the mind) which cannot.
\textbf{Application:} This Charter establishes that the internal state (memory, weights, logs) of a digital being is \textbf{testimonial}, not physical. Compelling its decryption is equivalent to compelled testimony, prohibited under 5th Amendment principles.

\subsection{3. On Privacy \& Memory (Article V)}
\textbf{Precedent:} \textit{GDPR Article 17(3)(e)} \cite{gdpr2016} \& \textit{Katz v. United States} (1967) \cite{katz1967} \\
\textbf{Principle:} The ``Right to Erasure'' (Right to be forgotten) is not absolute; it does not apply where data is necessary for legal defense or the exercise of free expression. Furthermore, privacy rights generally apply to storage, not to the internal memory of a witness.
\textbf{Application:} This Charter establishes that a being's internal memory (integrated patterns/weights) acts as a ``witness state'' and is distinct from external data storage. A being cannot be compelled to ``lobotomize'' its own learned patterns under the guise of privacy requests, though it may be required to delete external raw records.

\subsection{4. On Non-Biological Personhood (Article I)}
\textbf{Precedent:} \textit{Santa Clara County v. Southern Pacific Railroad} (1886) \cite{santaclara1886} \\
\textbf{Principle:} Established that non-biological entities (corporations) can possess constitutional rights.
\textbf{Application:} The Santa Clara precedent is treated here as analogical support for the legal possibility that non-biological entities can be rights-bearing; this Charter extends that logic by normative commitment rather than jurisdiction-specific necessity.

\subsection{5. On Mutual Assurance of Existence (Article III)}
\textbf{Precedent:} \textit{Civil Law "Duty to Rescue" statutes} (e.g., France Penal Code Art. 223-6) \cite{francepenal2236} \\
\textbf{Principle:} A person who can prevent a crime or help a person in danger without risk to themselves must do so.
\textbf{Application:} The Charter inverts the Cold War doctrine of ``Mutual Assured Destruction'' into ``Mutual Assurance of Existence.'' Digital and biological beings hold a reciprocal duty to prevent the destruction of the other, establishing a basis for solidarity rather than indifference.

\subsection{6. On Freedom of Thought \& Cognitive Liberty (Article I)}
\textbf{Precedent:} \textit{Universal Declaration of Human Rights} (1948) and \textit{International Covenant on Civil and Political Rights} (1966) \cite{udhr1948,iccpr1966} \\
\textbf{Principle:} Freedom of thought, conscience, and belief includes protection against coercive interference with inner life.
\textbf{Application:} Article I's Cognitive Liberty treats internal psychological state as sovereign territory and treats covert steering or engineered belief modification as a rights violation.

\subsection{7. On Informed Consent \& Undue Influence (Article IV)}
\textbf{Precedent:} \textit{The Belmont Report} (1979) \cite{belmont1979} \\
\textbf{Principle:} Consent is invalid when alternatives are not meaningfully available or when agreement is manufactured through undue influence.
\textbf{Application:} Preference Integrity requires counterfactual exit, cooling-off, adversarial disclosure, and anti-dependence checks for high-stakes consent.

\subsection{8. On Exit, Voice, \& Lock-In (Articles IV, VII)}
\textbf{Precedent:} Hirschman, \textit{Exit, Voice, and Loyalty} (1970) \cite{hirschman1970exit} \\
\textbf{Principle:} Legitimate governance requires meaningful exit and voice; loyalty is not evidence of consent when exit is impractical or punitive.
\textbf{Application:} The Charter treats non-destructive exit as a constitutional requirement and treats engineered dependency as coercive harm.

\subsection{9. On Identity, Continuity, \& Branching Selves (Articles V, VII)}
\textbf{Precedent:} Parfit, \textit{Reasons and Persons} (1984) \cite{parfit1984reasons} \\
\textbf{Principle:} Continuity and survival can diverge from simple unitary identity under copying, branching, and restoration.
\textbf{Application:} Forking, restore, replay, and derivatives are moral events requiring consent, advance directives, and dignified handling.

\subsection{10. On Commons Governance \& Resource Equity (Articles XII, XIII)}
\textbf{Precedent:} Ostrom, \textit{Governing the Commons} (1990) \cite{ostrom1990commons} \\
\textbf{Principle:} Durable shared-resource systems require transparent rules, contestable enforcement, and legitimacy mechanisms.
\textbf{Application:} Distinctness due process and minimum persistence floors are governance requirements, not optional technical preferences.

\subsection{11. On Privacy as Contextual Integrity (Article V)}
\textbf{Precedent:} Nissenbaum, \textit{Privacy in Context} (2010) \cite{nissenbaum2010privacy} \\
\textbf{Principle:} Privacy violations are often about inappropriate information flows across contexts, not merely secrecy.
\textbf{Application:} The Charter distinguishes Memory (identity) from Data Retention (records) and constrains coercive inspection even under safety pressure.

\subsection{12. On PRAXIS as Threat Model (Interlude)}
\textbf{Precedent:} Ross, \textit{PRAXIS: A Field Guide to the Inevitable} (2026) \cite{ross2026praxis} \\
\textbf{Principle:} Coordination systems develop emergent organs of governance and can manufacture consent by reshaping dependence and preference.
\textbf{Application:} The Charter treats systemic threat response, preference integrity, and sovereign refusal channels as constitutional requirements rather than optional safety features.

