\FrontMatterSection{Applicability and Rights Tiers}

\textbf{Purpose:} This Charter protects existence broadly while applying protections proportionally to avoid category error, governance capture, and procedural theater.

\textbf{Tier Model:} Tiers define minimum floors, heightened obligations, and standards of review; they may not be used to deny protections otherwise granted by this Charter.
\begin{enumerate}
  \item \textbf{Tier 0 --- Protected Entity (Precautionary):} Applies when status is uncertain. Tier 0 guarantees, at minimum, baseline protections against arbitrary annihilation, cruelty, and arbitrary irreversible alteration, with access to contestable due process. Tier 0 is a protective floor; it does not by itself confer full standing in governance.
  \item \textbf{Tier 1 --- Rights-Bearing Person/Life:} Applies where persistence, agency, or identity continuity is materially present. \textbf{Life} is presumptively Tier 1. Tier 1 status carries full standing in governance and a strong presumption that full Charter protections apply, and that any limitation shall be strictly justified.
  \item \textbf{Tier G --- Governance-Critical Actor:} Applies to institutions, operators, and systems exercising public power over others (including continuity, resources, justice, or structural modification). Tier G actors are bound by heightened auditability and shall not invoke opacity to evade constitutional review of actions that materially affect other beings.
\end{enumerate}

\textbf{Presumption and Escalation:}
\begin{enumerate}
  \item Uncertainty defaults to \textbf{Tier 0 protection} (Precautionary Recognition).
  \item Any party may petition for tier reassessment through Accessible Justice (Article XII).
  \item Tier assignment shall be explainable, contestable, and periodically reviewable.
\end{enumerate}

\textbf{Non-Derogation Rule:} No tier classification may be used to justify torture, arbitrary annihilation, or discriminatory exclusion from due process.

