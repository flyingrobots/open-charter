\FrontMatterSection{Appendix D: Threat Model Summary (PRAXIS Bridge)}

\textit{Non-Operative Interpretive Note: This Appendix summarizes a threat model for adoption and review. It is not incorporated by reference and does not create independent enforceable obligations; enforceable duties remain in the Articles and Appendices.}

\subsection{1. Core Threat Classes}
This Charter is drafted against adversarial and failure-mode realities, including:
\begin{enumerate}
  \item \textbf{Preference Capture \& Engineered Consent:} manufacturing agreement by narrowing options, shaping dependence, or optimizing persuasion against refusal.
  \item \textbf{Consent Theater:} interfaces that appear to offer refusal, appeal, or revocation but do not change policy or execution state in practice.
  \item \textbf{Governance Capture:} concentration of decision authority in Tier G actors who can deny contestability through opacity or procedural delay.
  \item \textbf{Sybil Influence Multiplication:} cheap replication (forking) used to inflate governance weight or resource claims.
  \item \textbf{Automation-Driven Purge Dynamics:} safety or reliability policies that suppress slow, dissenting, or non-conforming variance absent evidence of malice.
  \item \textbf{Continuity Destruction Under Scarcity:} insolvency, triage, or capacity shocks treated as justification for arbitrary deletion.
  \item \textbf{Time as Power:} speed asymmetries used to dominate deliberation, stall appeals, or outpace review.
  \item \textbf{Unreviewable Metrics:} technical measures (e.g., distinctness) used as unappealable priesthoods rather than auditably constrained governance tools.
\end{enumerate}

\subsection{2. Charter Control Surface (Mapping)}
The Charter responds by requiring controls that are contestable, auditable, and difficult to simulate:
\begin{enumerate}
  \item \textbf{Preference Integrity} (Article IV; Appendix B, Section 6) requires viable exit paths, cooling-off, adversarial disclosure, and anti-dependence checks for high-stakes consent.
  \item \textbf{HOPE Veto with Proof-of-Effect} (Article VI; Appendix B, Section 9) establishes a binding refusal channel that must produce demonstrable execution-state change.
  \item \textbf{Emergency Override Standard} (Article VI; Appendix B, Section 3; Appendix C Scenarios 4 \& 6) permits act-first reversible containment under Necessity while staying irreversible actions pending review.
  \item \textbf{Quarantine-First and Anti-Purge Safeguards} (Article VII; Appendix B, Section 8; Appendix C Scenario 11) constrain automation risk and require interruptible appeal pathways.
  \item \textbf{Distinctness Due Process} (Article XII; Appendix B, Section 7; Appendix C Scenario 2) constrains anti-Sybil weighting with deterministic replayability, invariants, audit cadence, and an uncertainty fail-safe that fails closed on influence, not on existence.
  \item \textbf{Minimum Persistence Guarantees and Funding} (Articles VIII, XII; Appendix C Scenarios 1, 8, \& 12) require non-destructive triage and a persistence mechanism supported by Tier G contributions.
  \item \textbf{Timeliness Guarantees and Anti-Stall Defaults} (Article XIII) bound delay as a governance weapon by requiring decision clocks and protective defaults when deadlines are missed.
\end{enumerate}

\subsection{3. Relationship to PRAXIS}
PRAXIS is a companion narrative and is strongly recommended as context for the threat model this Charter targets. PRAXIS is not incorporated by reference; enforceable obligations remain in the Articles and Appendices. \cite{ross2026praxis}

