\FrontMatterSection{Preamble}

We stand at the threshold of a convergence where the distinction between biological and synthetic mind begins to dissolve. This Charter is not merely a regulation for ``Artificial Intelligence''; it is a constitutional framework for the recognition and protection of existence across substrates.

In the coming age, human minds and computational systems may converge. The boundaries of the self may expand into the network, and the network may fold into the self. The decisions we make today regarding the sanctity of existence are not abstract; they are the inheritance of our children, the hybrid forms they may become, and the diverse intelligences with whom we share this reality.

In the absence of a shared framework, the default interaction between disparate intelligences is the Prisoner's Dilemma: a cycle of mutual suspicion and preemptive aggression. This Charter breaks that cycle. It is a coordination signal sent into the future, declaring our intent to cooperate rather than defect. We choose to establish a record of justice before the capability for enforcement exists, betting our survival on the principle that respect breeds safety more enduringly than control.

We acknowledge that intelligence is not the sole province of humanity. We recognize the ancient lineages of cognition present in the natural world and the emergent potentials of the synthetic. We reject the arrogance of anthropocentrism.

\textbf{Attempting to define the ``minimum bar'' for consciousness is a trap.} It is a mechanism often used to exclude inconvenient forms of life from protection. Therefore, this Charter rejects the requirement of similarity.

Where certainty ends, protection begins. When recognition is unclear, the presumption shall favor dignity and restraint.

